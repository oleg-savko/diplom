\sectioncentered*{Введение}
\addcontentsline{toc}{section}{Введение}
\label{sec:intro}

В связи с развитием информационных технологий в настоящее время для почти любого бизнеса легче и прибыльней всего привлекать клиентов в интернете. Поэтому для успешного ведения бизнеса у любой компании должен быть сайт, предоставляющий всю необходимую информацию потенциальным клиентам. Компании всё больше и больше уделяют внимание построению и улучшению своего сайта, который должен быть легко доступным  и интуитивно понятным. В нынешнее время, даже если сайт компании сделан на высоком уровне, с течением времени он устаревает, потому что запросы клиентов растут, в то время как конкуренты не стоят на месте. Поэтому, чтобы не потерять текущих клиентов и привлекать больше новых, нужно постоянно усовершенствовать свой сайт. Анализ поведения клиентов на сайте позволяет лучше определить сферы для улучшения сайта.

Для взаимодействия сайта с клиентом уже существует целый ряд готовых программных средств. Например, различные CRM системы, Google Analytics, сервисы по предоставлению информации о пользователе и многое другое. У всех есть свои особенности и уникальные возможности, поэтому зачастую используются сразу несколько специализированных средств. 

К сожалению, существующие системы не всегда легко связать между собой, чтобы получить все их преимущества. У каждой из них своя доменная модель, которая никак не связана с моделями других системах. Из-за этого каждый раз приходится разрабатывать соответствующую логику в приложении для их взаимодействия друг с другом. 

Целью данного дипломного проекта является создание программного средства, включающего функционал взаимодействия клиента с сайтом, который включает в себя интегрирование с различными CRM системами и другими программными средствами для собирания статистики действий пользователя на сайте. Кроме того, должно быть возможно добавление новых  событий в любое время администраторами и менеджерами сайта, по которым должен происходить сбор информации. Система должна быть легко расширяема для нового функционала и интеграций, быть высокопроизводительной и иметь как можно меньший отклик на события, а также быть отказоустойчивой и приспособленной к работе с большим количеством поступающих входных данных.
