\begin{titlepage}
  \begin{center}
    Министерство образования Республики Беларусь\\[1em]
    Учреждение образования\\
    БЕЛОРУССКИЙ ГОСУДАРСТВЕННЫЙ УНИВЕРСИТЕТ \\
    ИНФОРМАТИКИ И РАДИОЭЛЕКТРОНИКИ\\[1em]

    \begin{minipage}{\textwidth}
      \begin{flushleft}
        \begin{tabular}{ l }
          Факультет компьютерных систем и сетей\\
          Кафедра программного обеспечения информационных технологий
        \end{tabular}
      \end{flushleft}
    \end{minipage}\\[1em]

    \begin{flushright}
      \begin{minipage}{0.4\textwidth}
        \textit{К защите допустить:}\\[0.8em]
        Заведующий кафедрой ПОИТ\\[0.45em]
        \underline{\hspace*{2.8cm}} Н.\,В.~Лапицкая
      \end{minipage}\\[2.2em]
    \end{flushright}

    %%
    %% ВНИМАНИЕ: на некторых факультетах (ФКП) и кафедрах (ПИКС) слова "ПОЯСНИТЕЛЬНАЯ ЗАПИСКА" предлагается (требуется) оформлять полужирным начертанием. Раскомментируйте нужную для вас строку:
    %%
    %\textbf{ПОЯСНИТЕЛЬНАЯ ЗАПИСКА}\\
    {ПОЯСНИТЕЛЬНАЯ ЗАПИСКА}\\
    {к дипломному проекту}\\
    {на тему}\\[1em]
    \textbf{\large ИНФОРМАЦИОННО-АНАЛИТИЧЕСКОЕ ПРОГРАММНОЕ СРЕДСТВО ВЗАИМОДЕЙСТВИЯ КЛИЕНТА С САЙТОМ}\\[1em]


    {БГУИР ДП 1-40 01 01 03 088 ПЗ}\\[2em]

    \begin{tabular}{ p{0.65\textwidth}p{0.25\textwidth} }
      Студент & О.\,А.~Савко \\
      Руководитель & О.\,А.~Мацкевич \\
      Консультанты: &\\
      \hspace*{3ex}\emph{от кафедры ПОИТ} & И.\,С.~Король \\
      \hspace*{3ex}\emph{по экономической части} & К.\,Р.~Литвинович \\
      %%
      %% ВНИМАНИЕ: в зависимости от выбранной темы, у вас консультант может быть как по охране труда, так и по:
        % экологической безопасности
        % ресурсосбережению
        % энергосбережению
      %%
      %% Впишите правильную формулировку по необходимости
      Нормоконтролёр & С.\,В.~Болтак \\
      & \\
      Рецензент &
    \end{tabular}

    \vfill
    {\normalsize Минск 2016}
  \end{center}
\end{titlepage}
