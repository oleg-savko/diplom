\section{Тестирование приложения}
\label{sec:testing}


\subsection{Тестирование панели администратора}
Для оценки правильности работы функциональных требований в панели администратора было проведено тестирование, включающие тест кейсы представленные      в таблице~\ref{table:testing:admin} 
  
\begin{longtable}[l]{| >{\raggedright}m{0.3\textwidth}
                  | >{\raggedright}m{0.4\textwidth}
                  | >{\raggedright\arraybackslash}m{0.2\textwidth}|}
  \caption{Тестирование микросервиса панели администратора}
  \label{table:testing:admin} \tabularnewline


    \hline
    Название тест-кейса и его описание & Ожидаемый результат  & Полученный результат \\
    \hline
    \centering{1} & \centering{2} & \centering{3} \tabularnewline
    \hline

    Авторизация в панель администратора \\
    a) Зайти на страницу входа в аккаунт панель администратора \\
    б) Ввести логин и пароль \\
    в) Нажать кнопку Login 

    & 

    a) Отображается страница входа в аккаунт панель администратора \\
    б) Необходимые поля доступны для заполнения \\
    в) Вход в панель администратора

    & 

    Пройден \\ \hline

    % Создание нового события
    Создание нового события \\
    a) Перейти на страницу создания нового события \\
    б) Ввести обязательные поля \\
    в) Нажать кнопку Create 

    & 

    a) Отображается страница создания нового события \\
    б) Необходимые поля доступны для заполнения \\
    в) Событие успешно создано 
    
    & 

    Пройден \\ 

    \pagebreak
    \caption*{Продолжение таблицы~\ref{table:testing:admin}} \\
    \hline
    \centering 1 & \centering 2 & \centering 3 \tabularnewline
    \hline


    % Добавление поля для события
    Добавление поля для события\\
    a) Выбрать события \\
    б) Нажать кнопку Add Prop \\
    в) Ввести обязательные поля \\
    г) Нажать кнопку Add 

    & 

    a) Отображается страница события \\
    б) Отображается popup добавления поля события \\
    в) Необходимые поля доступны для заполнения \\
    г) Поле успешно добавлено в событие 
    
    & 

    Пройден \\ \hline

    Просмотр онлайн статистики поступление событий \\
    а) Зайти в панель администратора \\
    б) Перейти на страницу статистики 
    
    & 

    а) Отображается главной страницы  \\
    б) Отображается график текущих поступающими событиями, который изменять с течение времени без перезагрузки страницы 
    
    & 

    Пройден \\

    % \pagebreak
    % \caption*{Продолжение таблицы~\ref{table:testing:admin}} \\
    % \hline
    % \centering 1 & \centering 2 & \centering 3 \tabularnewline
    \hline


    % Просмотр статисти поступивших событий за промежуток времени
    Просмотр статистики поступивших событий за промежуток времени \\
    a) Зайти в панель администратора \\
    б) Перейти на страницу статистики \\
    в) Выбрать интервал времени \\
    г) Нажать кнопку Show 
    
    & 

    a) Отображается главной страницы  \\
    б) Отображается график текущих поступающими событиями, который изменять с течение времени без перезагрузки страницы \\
    в) Необходимые поля доступны для заполнения \\
    г) Отображается график поступивших событий за выбранный промежуток 
    
    & 

    Пройден \\ \hline

  \hline

\end{longtable}


Таким образом результат тестирования подтверждает, что микросервис ответственный за панель администратора работает корректно с установленными требованиями.

\subsection{Тестирование интеграции с Salesforce CRM}
Для оценки правильности работы функциональных требований связанных с интеграцией salesforce было проведено тестирование, включающие тест кейсы представленные в таблице~\ref{table:testing:sf} 


\begin{longtable}[l]{| >{\raggedright}m{0.3\textwidth}
                  | >{\raggedright}m{0.3\textwidth}
                  | >{\raggedright\arraybackslash}m{0.3\textwidth}|}
  \caption{Тестирование микросервиса интеграции с Salesforce CRM}
  \label{table:testing:sf} \tabularnewline

  \hline
       Название тест-кейса и его описание & Ожидаемый результат  & Полученный результат \\
    \hline
    \centering{1} & \centering{2} & \centering{3} \tabularnewline
    \hline

   % Тестирование онлайн чата
    Тестирование настроек онлайн чата в sf \\
    a) Агент заходит в Salesforce CRM \\
    б) Агент настраивает параметры в salesforce для онлайн чата (специфика sf) \\
    в) Агент настраивает страницу сайта клиента, на которой должен появляться чат с пользователем \\
    г) Агент заходить Salesforce Console
    & 
    a) Отображается главной страница Salesforce CRM \\
    б) Параметры доступны для настройки \\
    в) Необходимые поля доступны для заполнения \\
    г) Отображается Salesforce Console доступна
    & 
    Пройден \\ 

    \pagebreak
    \caption*{Продолжение таблицы~\ref{table:testing:admin}} \\
    \hline
    \centering 1 & \centering 2 & \centering 3 \tabularnewline
   
    \hline

    Тестирование онлайн чата \\
    a) Агент ставит в Salesforce Console себе статус онлайн \\
    б) Пользователь заходит на страницу сайта, на которой настроен онлайн чат \\
    в) Выполняются условия по котором должен показаться онлайн чат \\
    г) Пользователь нажимает кнопку Start Chat \\
    д) Агент нажимает кнопку принять чат \\
    е) Агент отправляет сообщение \\
    ж) Пользователь отправляет сообщение \\
    з) Пользователь закрывает чат 
    
    & 

    a) Агенту проставляется статус онлайн \\
    б) Страница сайта корректно загружается \\
    в) На странице всплывает окошко для онлайн чата \\
    г) Открывается окно ожидание онлайн чата с агентом и у агента в Salesforce Console отображается входящий вызов  \\
    д) Открывается чат с агентом \\
    е) Пользователь видит сообщение \\
    ж) Агент видит сообщение \\
    з) Чат успешно закрывается 
    
    & 
    
    Пройден \\ \hline

\end{longtable}
  
Таким образом результат тестирования подтверждает, что микросервис ответственный за интеграцию с salesforce работает корректно с установленными требованиями.

\subsection{Тестирование отказоустойчивости}
Для оценки работы системы в непредвиденных ситуациях (например проблемы с сетью, проблемы у интернет провайдера или вообще сгорел сервер) было проведено тестирование отказоустойчивости системы, включающие тест кейсы представленные в таблице~\ref{table:testing:fb} 
\begin{longtable}[l]{| >{\raggedright}m{0.3\textwidth}
                  | >{\raggedright}m{0.3\textwidth}
                  | >{\raggedright\arraybackslash}m{0.3\textwidth}|}
  \caption{Тестирование отказоустойчивости системы}
  \label{table:testing:fb} \tabularnewline

  \hline
       Название тест-кейса и его описание & Ожидаемый результат  & Полученный результат \\
    \hline
    \centering{1} & \centering{2} & \centering{3} \tabularnewline
    \hline


    Тестирование отказоустойчивости падения одного процесса микросервиса \\
    а) Запустить 2 экземпляра микросервиса ответственного за интеграцию с salesforce
    б) Администратор заходит в Hystrix Dashboard \\
    в) Администратор вводит соответствующую ссылку /turbine/turbine.stream в поле для заполнение и нажимает кнопку Monitor Stream \\
    г) С эмулировать падение процесса сервиса: на сервере убить процесс микросервиса ответственного за интеграцию с salesforce с помощью команды kill -9 PID (PID - id процесса) \\
    & 
    Тестирование отказоустойчивости падения одного процесса микросервиса \\
    а) Поднялась 2 процесса микросервиса ответственного за интеграцию с salesforce \\
    б) Отображается главная страница Hystrix Dashboard \\
    в) Отображается dashboard с real-time графиками состояний статусов происходящих событий в системе \\
    г) Система по-прежнему работает \\
    & 
    Пройден \\ 

    \pagebreak
    \caption*{Продолжение таблицы~\ref{table:testing:fb}} \\
    \hline
    \centering 1 & \centering 2 & \centering 3 \tabularnewline
   
    \hline


    % Тестирование отказоустойчивости
    Тестирование отказоустойчивости падения всех процессов микросервиса \\
    a) Администратор заходит в Hystrix Dashboard \\
    б) Администратор вводит соответствующую ссылку /turbine/turbine.stream в поле для заполнения и нажимает кнопку Monitor Stream \\
    в) С эмулировать падение сервиса: на сервере последовательно аварийно завершить все процессы микросервиса, ответственного за интеграцию с salesforce с помощью команды kill -9 PID (PID - id процесса) \\
    г) Запустить микросервис ответствененный за интеграцию с salesforce
    & 
    Тестирование отказоустойчивости падения всех процессов микросервиса \\
    a) Отображается главная страница Hystrix Dashboard \\
    б) Отображается dashboard с real-time графиками состояний статусов происходящих событий в системе \\
    в) На графики интеграции с salesforce разомкнулся Circle Breaker, вследствие чего график стал красным и запросы перестали отправляться в сервис интеграции. Вместо этого стали выполняться fallback методы \\
    г) Circle Breaker замкнулся, вследствие чего запросы опять стали ходить в микросервис интеграции с salesforce
    & 
    Пройден \\ \hline

\end{longtable}

